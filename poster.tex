\documentclass[portrait]{baposter}

\usepackage{times}
\usepackage{calc}
\usepackage{graphicx}
\usepackage{amsmath}
\usepackage{amssymb}
\usepackage{relsize}
\usepackage{multirow}
\usepackage{bm}
\usepackage{minted}
\usepackage{booktabs}

\usepackage{graphicx}
\usepackage{multicol}

\usepackage{pgfbaselayers}
\pgfdeclarelayer{background}
\pgfdeclarelayer{foreground}
\pgfsetlayers{background,main,foreground}

\newcommand{\captionfont}{\footnotesize}

\selectcolormodel{cmyk}

\graphicspath{{images/}}

%%%%%%%%%%%%%%%%%%%%%%%%%%%%%%%%%%%%%%%%%%%%%%%%%%%%%%%%%%%%%%%%%%%%%%%%%%%%%%%%
%%%% Some math symbols used in the text
%%%%%%%%%%%%%%%%%%%%%%%%%%%%%%%%%%%%%%%%%%%%%%%%%%%%%%%%%%%%%%%%%%%%%%%%%%%%%%%%

%%%%%%%%%%%%%%%%%%%%%%%%%%%%%%%%%%%%%%%%%%%%%%%%%%%%%%%%%%%%%%%%%%%%%%%%%%%%%%%%
% Multicol Settings
%%%%%%%%%%%%%%%%%%%%%%%%%%%%%%%%%%%%%%%%%%%%%%%%%%%%%%%%%%%%%%%%%%%%%%%%%%%%%%%%
\setlength{\columnsep}{0.5em}
\setlength{\columnseprule}{0mm}

%%%%%%%%%%%%%%%%%%%%%%%%%%%%%%%%%%%%%%%%%%%%%%%%%%%%%%%%%%%%%%%%%%%%%%%%%%%%%%%%
% Save space in lists. Use this after the opening of the list
%%%%%%%%%%%%%%%%%%%%%%%%%%%%%%%%%%%%%%%%%%%%%%%%%%%%%%%%%%%%%%%%%%%%%%%%%%%%%%%%
\newcommand{\compresslist}{
    \setlength{\itemsep}{1pt}
    \setlength{\parskip}{0pt}
    \setlength{\parsep}{0pt}
}

%%%%%%%%%%%%%%%%%%%%%%%%%%%%%%%%%%%%%%%%%%%%%%%%%%%%%%%%%%%%%%%%%%%%%%%%%%%%%%
%%% Begin of Document
%%%%%%%%%%%%%%%%%%%%%%%%%%%%%%%%%%%%%%%%%%%%%%%%%%%%%%%%%%%%%%%%%%%%%%%%%%%%%%

\begin{document}

%%%%%%%%%%%%%%%%%%%%%%%%%%%%%%%%%%%%%%%%%%%%%%%%%%%%%%%%%%%%%%%%%%%%%%%%%%%%%%
%%% Here starts the poster
%%%---------------------------------------------------------------------------
%%% Format it to your taste with the options
%%%%%%%%%%%%%%%%%%%%%%%%%%%%%%%%%%%%%%%%%%%%%%%%%%%%%%%%%%%%%%%%%%%%%%%%%%%%%%
\typeout{Poster Starts}
\background{
  \begin{tikzpicture}[remember picture,overlay]
    \draw (current page.north west)+(-2em,-0em) node[anchor=north west] {\hspace{-2em}\includegraphics[height=1.1\textheight]{background}};
  \end{tikzpicture}
}
\definecolor{silver}{cmyk}{0,0,0,0.3}
\definecolor{yellow}{cmyk}{0,0,0.9,0.0}
\definecolor{reddishyellow}{cmyk}{0,0.22,1.0,0.0}
\definecolor{black}{cmyk}{0,0,0.0,1.0}
\definecolor{darkYellow}{cmyk}{0,0,1.0,0.5}
\definecolor{darkSilver}{cmyk}{0,0,0,0.1}

\definecolor{lightyellow}{cmyk}{0,0,0.3,0.0}
\definecolor{lighteryellow}{cmyk}{0,0,0.1,0.0}
\definecolor{lighteryellow}{cmyk}{0,0,0.1,0.0}
\definecolor{lightestyellow}{cmyk}{0,0,0.05,0.0}
\begin{poster}{
  % Show grid to help with alignment
  grid=no,
  % Column spacing
  colspacing=1em,
  % Color style
  bgColorOne=lighteryellow,
  bgColorTwo=lightestyellow,
  borderColor=reddishyellow,
  headerColorOne=yellow,
  headerColorTwo=reddishyellow,
  headerFontColor=black,
  boxColorOne=lightyellow,
  boxColorTwo=lighteryellow,
  % Format of textbox
  textborder=rectangle,
  % Format of text header
  eyecatcher=no,
  headerborder=open,
  headerheight=0.08\textheight,
  headershape=roundedright,
  headershade=plain,
  headerfont=\Large\textsf, %Sans Serif
  boxshade=plain,
%  background=shade-tb,
  background=plain,
  linewidth=2pt
  }
  % Eye Catcher
  {} % No eye catcher for this poster. If an eye catcher is present, the title is centered between eye-catcher and logo.
  % Title
  {\sf %Sans Serif
  %\bf% Serif
  Review of Python-based symbolic mathematics systems and libraries}
  % Authors
  {\sf %Sans Serif
  % Serif
  Mateusz Paprocki$^{1,2}$
  \hspace{3em}
  $^1$ University of Nevada, Reno,
  $^2$ SymPy Development Team
  }
  % University logo
  {{\begin{minipage}{12em}
    \hfill
    \includegraphics[height=8.0em]{sympy-logo}
  \end{minipage}}
  }

  \tikzstyle{light shaded}=[top color=baposterBGtwo!30!white,bottom color=baposterBGone!30!white,shading=axis,shading angle=30]

  % Width of left inset image
     \newlength{\leftimgwidth}
     \setlength{\leftimgwidth}{0.78em+8.0em}

%%%%%%%%%%%%%%%%%%%%%%%%%%%%%%%%%%%%%%%%%%%%%%%%%%%%%%%%%%%%%%%%%%%%%%%%%%%%%%
%%% Now define the boxes that make up the poster
%%%---------------------------------------------------------------------------
%%% Each box has a name and can be placed absolutely or relatively.
%%% The only inconvenience is that you can only specify a relative position
%%% towards an already declared box. So if you have a box attached to the
%%% bottom, one to the top and a third one which should be in between, you
%%% have to specify the top and bottom boxes before you specify the middle
%%% box.
%%%%%%%%%%%%%%%%%%%%%%%%%%%%%%%%%%%%%%%%%%%%%%%%%%%%%%%%%%%%%%%%%%%%%%%%%%%%%%

%%% {name=hpfem,column=0,below=motivation,span=1,above=bottom}
%%% {name=motivation,column=0,below=contribution}

%%%%%%%%%%%%%%%%%%%%%%%%%%%%%%%%%%%%%%%%%%%%%%%%%%%%%%%%%%%%%%%%%%%%%%%%%%%%%%
\headerbox{Abstract}{name=abstract,column=0,row=0}{
%%%%%%%%%%%%%%%%%%%%%%%%%%%%%%%%%%%%%%%%%%%%%%%%%%%%%%%%%%%%%%%%%%%%%%%%%%%%%%
x\\x\\x\\x\\x
}

%%%%%%%%%%%%%%%%%%%%%%%%%%%%%%%%%%%%%%%%%%%%%%%%%%%%%%%%%%%%%%%%%%%%%%%%%%%%%%
\headerbox{Why reinvet the wheel?}{name=why,column=1,row=0}{
%%%%%%%%%%%%%%%%%%%%%%%%%%%%%%%%%%%%%%%%%%%%%%%%%%%%%%%%%%%%%%%%%%%%%%%%%%%%%%
There are numerous symbolic manipulation systems:
Proprietary software: Mathematica, Maple, Magma, ...
Open Source software: AXIOM, GiNaC, Maxima, PARI, Sage, Singular, Yacas, ...
}

%%%%%%%%%%%%%%%%%%%%%%%%%%%%%%%%%%%%%%%%%%%%%%%%%%%%%%%%%%%%%%%%%%%%%%%%%%%%%%
\headerbox{Why we chose Python?}{name=why,column=2,row=0}{
%%%%%%%%%%%%%%%%%%%%%%%%%%%%%%%%%%%%%%%%%%%%%%%%%%%%%%%%%%%%%%%%%%%%%%%%%%%%%%
Python is modern general purpose programming language that is easy to learn
but is also very expressive. It's widely used (Google, NASA, ...).
huge number of libraries
numerical computation: NumPy, SciPy
physics, simulation, bioinformatics
visualisation, 3D graphics, plotting
databases, networking, . . .
easy to bind with other code
C/C++ via native API or Cython
Fortran using f2py bindings
}

%%%%%%%%%%%%%%%%%%%%%%%%%%%%%%%%%%%%%%%%%%%%%%%%%%%%%%%%%%%%%%%%%%%%%%%%%%%%%%
\headerbox{Facts}{name=facts,column=0,span=3,below=abstract}{
%%%%%%%%%%%%%%%%%%%%%%%%%%%%%%%%%%%%%%%%%%%%%%%%%%%%%%%%%%%%%%%%%%%%%%%%%%%%%%
\begin{tabular}{l|lllll}
                & Sage                         & Swiginac              & SymPy                & SymPyCore                   & Pynac              \\\midrule
Website         & www.sagemath.org             & swiginac.berlios.de   & www.sympy.org        & code.google.com/p/sympycore & pynac.sagemath.org \\
Initial release & 24 February 2005             & 25 September 2005     & 11 March 2007        & 29 February 2008            & 8 August 2008      \\
Stable release  & 4.7 (23 May 2011)            & 1.5.1 (15 April 2009) & 0.7.1 (29 July 2011) & 0.1 (29 February 2008)      & 0.2.2 (9 May 2011) \\
Download size   & 411 MB (Ubuntu 64-bit, lzma) & 100 KB (gzip)         & 3.4 MB (gzip)        & 138 KB (gzip)               & 2.1 MB (bzip2)     \\
Installed size  & 2.4 GB                       & very little           & very little          & very little                 & very little        \\
License         & GNU GPL                      & GNU GPL               & New BSD              & New BSD                     & GNU GPL            \\
Author          & William Stein                & Ola Skavhaug          & Ondrej Certik        & Pearu Peterson              & Burcin Erocal      \\
Contributors    & $>150$                       & $7$                   & $>120$               & $3$                         & $7$                \\
SCM software    & Mercurial (HG)               & SVN                   & GIT                  & SVN                         & Mercurial (HG)     \\
Languages       & Python, Cython               & C++, Python           & Python               & Python, C                   & Cython, Python     \\
Web interface   & www.sagenb.org               & none                  & live.sympy.org       & none                        & www.sagenb.org     \\
Standalone      & yes                          & yes                   & yes                  & yes                         & no (Sage)          \\
\end{tabular}
}

%%%%%%%%%%%%%%%%%%%%%%%%%%%%%%%%%%%%%%%%%%%%%%%%%%%%%%%%%%%%%%%%%%%%%%%%%%%%%%
\headerbox{Descriptions}{name=desc,column=0,span=2,below=facts}{
%%%%%%%%%%%%%%%%%%%%%%%%%%%%%%%%%%%%%%%%%%%%%%%%%%%%%%%%%%%%%%%%%%%%%%%%%%%%%%
{\bf Sage} is a free open-source mathematics software system licensed under the GPL. It
combines the power of many existing open-source packages into a common Python-based interface.
\\\\
{\bf Swiginac} is a Python interface to GiNaC, built with SWIG. The aim of swiginac is
to make all the functionality of GiNaC accessible from Python as an extension module.
\\\\
{\bf SymPy} is a Python library for symbolic mathematics. It aims to become a full-featured
computer algebra system (CAS) while keeping the code as simple as possible in order to be
comprehensible and easily extensible. SymPy is written entirely in Python and does not require
any external libraries.
\\\\
{\bf SymPyCore} is inspired by many attempts to implement CAS for Python and it is created
to fix SymPy performance and robustness issues. Sympycore does not yet have nearly as many
features as SymPy. Our goal is to work on in direction of merging the efforts with the SymPy
project in the future.
\\\\
{\bf Pynac} is a derivative of the C++ library GiNaC, which allows manipulation of symbolic
expressions. It currently provides the backend for symbolic expressions in Sage. The main
difference between Pynac and GiNaC is that Pynac relies on Sage to provide the operations
on numerical types, while GiNaC depends on CLN for this purpose.
}

%%%%%%%%%%%%%%%%%%%%%%%%%%%%%%%%%%%%%%%%%%%%%%%%%%%%%%%%%%%%%%%%%%%%%%%%%%%%%%
\headerbox{Examples}{name=examples,column=2,span=1,below=facts}{
%%%%%%%%%%%%%%%%%%%%%%%%%%%%%%%%%%%%%%%%%%%%%%%%%%%%%%%%%%%%%%%%%%%%%%%%%%%%%%
{\small
\inputminted{python}{code/sage.py}
\inputminted{python}{code/swiginac.py}
\inputminted{python}{code/sympy.py}
\inputminted{python}{code/sympycore.py}
}
}

%%%%%%%%%%%%%%%%%%%%%%%%%%%%%%%%%%%%%%%%%%%%%%%%%%%%%%%%%%%%%%%%%%%%%%%%%%%%%%
\headerbox{Sage vs. SymPy}{name=sagesympy,column=0,span=1,below=desc}{
%%%%%%%%%%%%%%%%%%%%%%%%%%%%%%%%%%%%%%%%%%%%%%%%%%%%%%%%%%%%%%%%%%%%%%%%%%%%%%
Although Sage's and SymPy's syntax is very similar, those are very different
systems.
}

%%%%%%%%%%%%%%%%%%%%%%%%%%%%%%%%%%%%%%%%%%%%%%%%%%%%%%%%%%%%%%%%%%%%%%%%%%%%%%
\headerbox{SymPy vs. SymPyCore}{name=sympysympycore,column=1,span=1,below=desc}{
%%%%%%%%%%%%%%%%%%%%%%%%%%%%%%%%%%%%%%%%%%%%%%%%%%%%%%%%%%%%%%%%%%%%%%%%%%%%%%
SymPyCore is a fork and extension of \texttt{sympy.core} module.
}

%%%%%%%%%%%%%%%%%%%%%%%%%%%%%%%%%%%%%%%%%%%%%%%%%%%%%%%%%%%%%%%%%%%%%%%%%%%%%%
\headerbox{Pynac vs. Swiginac}{name=pynacswiginac,column=2,span=1,below=desc}{
%%%%%%%%%%%%%%%%%%%%%%%%%%%%%%%%%%%%%%%%%%%%%%%%%%%%%%%%%%%%%%%%%%%%%%%%%%%%%%
Pynac and Swiginac both wrap Ginac, a C++ library for symbolic computing. Pynac
uses Cython to wrap C++ code and Swiginac uses SWIG library. Both don't wrap the
entire library. Pynac requires Sage for coefficient arithmetics, so you can't use
it a standalone library, as opposed to Swiginac.
}

%%%%%%%%%%%%%%%%%%%%%%%%%%%%%%%%%%%%%%%%%%%%%%%%%%%%%%%%%%%%%%%%%%%%%%%%%%%%%%
\headerbox{Conclusions}{name=conclusions,column=2,span=1,above=bottom}{
%%%%%%%%%%%%%%%%%%%%%%%%%%%%%%%%%%%%%%%%%%%%%%%%%%%%%%%%%%%%%%%%%%%%%%%%%%%%%%
If you are looking for a ready to use mathematics system that is supposed to
replace Mathematica or Maple, then Sage is probably the optimal choice among
Python-base mathematics. Otherwise, if you need a small, possibly embeddable,
library, then SymPy is a better choice.
}

\end{poster}
\end{document}
